\chapter{Introduction and Background}

Electron repulsion integrals (ERIs) and operations involving them pose a fundamental
computational problem for quantum chemistry. These 2-electron, 4-center integrals take the form:
\begin{align}(\mu \nu | \lambda \sigma) = \int \int \mu(\textbf{r}_{1}) 
\nu(\textbf{r}_{1}) r^{-1}_{12} \lambda(\textbf{r}_{2}) \sigma(\textbf{r}_{2}) 
d{\textbf{r}_{1}}d{\textbf{r}_{2}},
\end{align}

\noindent where $\mu, \nu, \lambda, \sigma$ are atomic orbital (AO) basis functions and $\textbf{r}_{1}$ and $\textbf{r}_{2}$
are electron coordinates \cite{ref1}. For simplicity, we have assumed the orbitals are real functions, as is usually
the case in practice. Immediately, one should note the $\mathcal{O}(N_{AO}^4)$ storage costs (where $N_{AO}$ is
the number of AO basis functions), which quickly 
limits the size of in-core investigations. Moreover, various fundamental operations involving ERIs 
are also costly. Here we focus on two types of operations. 
The first is the construction of the Coulomb and exchange matrices:
\begin{align}
J_{\mu \nu} &= (\mu \nu|\lambda \sigma)D_{\lambda \sigma} \\
K_{\mu \nu} &= (\mu \lambda|\nu \sigma)D_{\lambda \sigma},
\end{align}

\noindent where $J_{\mu \nu}$, $K_{\mu \nu}$, 
and $D_{\lambda \sigma}$ are the Coulomb, exchange, and density matrices, respectively.
Note that the use of the Einstein summation notation implies summations over indices.
Evaluating the 4-center ERIs according to (1.2) and (1.3) requires $\mathcal{O}(N_{AO}^4)$ operations, 
which is a heavy computational burden that plagues even the most basic quantum chemistry procedures.
For example, the Fock matrix for Restricted Hartree-Fock is built in the AO basis as:
\begin{align}
F_{\mu \nu} = H_{\mu \nu} + 2J_{\mu \nu} - K_{\mu \nu}
\end{align} 

\noindent Hartree-Fock is the simplest quantum chemistry procedure, also known as the Self-Consistent Field approach, 
because it involves iteratively
reconstructing the Fock matrix until convergence. Except for very large systems with large degrees of spatial sparsity, 
the most computationally demanding operation for this
method is the evaluation of the Coulomb and exchange matrices. 
 
The second fundamental operation considered here involves the transformation of the ERIs from the AO basis 
to the molecular orbital (MO) basis. This operation is written as:
\begin{align}
(pq | rs) = C_{\mu p}C_{ \nu q}
(\mu \nu | \lambda \sigma)C_{\lambda r}C_{\sigma s},
\end{align} 

\noindent where $p,q,r,s$ denote MO indices and each MO is expanded as a linear combination of AO's as 
$\phi_p(\textbf{r}_{2})=\sum_\mu C_{\mu p} \phi_p(\textbf{r}_{2})$.
These integral transformations scale 
as $\mathcal{O}(N_{AO}^4N_p)$ (where $N_p$ is the size of the smallest subset of MO's required in the target integrals). 
The transform serves as a major 
bottleneck for perturbative correlation methods like second-order M{\o}ller-Plesset perturbation theory (MP2).

To overcome the drastic cost of evaluation, transforming, and processing the 4-center ERIs, various approximations 
and screening techniques have been developed. 
Density fitting \cite{Whitten:1973:4496, Dunlap:1977:81, Dunlap:1979:3396, Dunlap:1979:4993, Kendall:1997:158}
is one method for alleviating these costs that has been gaining popularity in electronic structure theory 
\cite{Sherrill:2010:110902, Feyereisen:1993:359, Weigend:2002:4285, ref4, ref5}.
By utilizing the resolution of the identity technique, 
density fitting reduces the rank of the 4-center ERIs and approximates them using a 3-center form: 
\begin{align} 
(\mu \nu|P) &= \int \int \mu(\textbf{r}_{1}) 
\nu(\textbf{r}_{1}) r^{-1}_{12} P(\textbf{r}_{2}) d\textbf{r}_{1} d\textbf{r}_{2} \\
[J]_{PQ} &= \int \int P(\textbf{r}_1)r_{12}^{-1}Q(\textbf{r}_2) d{\textbf{r}_{1}}d{\textbf{r}_{2}} \\
(\mu \nu|\lambda \sigma) &\approx (\mu \nu| P)[J]_{PQ}^{-1}(Q|\lambda \sigma)  
\end{align} 

\noindent Here we have introduced an auxiliary basis $P, Q$ and the Coulomb fitting metric $[J]_{PQ}$ \cite{ref2}. 
In practice, permuational symmetry is almost always used so that only one set of these 3-center integrals need be built:
\begin{align} 
B_{\mu \nu}^P &= [J]_{PQ}^{-\frac{1}{2}}(\mu \nu | Q)  \\ 
(\lambda \sigma | \mu \nu) &\approx  B_{\mu \nu}^P B_{\lambda \sigma}^P  \end{align} 

\noindent In this work, tensors labeled $B$ indicate integrals that have been contracted with the fitting metric.
The rank-reduced 3-center ERIs have a storage requirement of $\mathcal{O}(N_{aux}N_{AO}^2)$, 
which allows many more investigations to be run in-core compared to the original 4-center ERIs. 
Using the 3-center ERIs, the Coulomb and exchange matrices are evaluated as: 
\begin{align}
J_{\mu \nu} &= B_{\mu \nu}^P B_{\lambda \sigma}^PD_{\lambda \sigma} \\
K_{\mu \nu} &= B_{\mu \lambda}^P B_{\nu \sigma}^PD_{\lambda \sigma}
\end{align}

\noindent Using (1.11), the Coulomb matrix is evaluated in $\mathcal{O}(N_{aux}N_{AO}^2)$ operations, which is a great success
for reducing computational complexity. On the other hand, (1.12) still requires $\mathcal{O}(N_{aux}N_{AO}^3)$ operations. 
Since $N_{aux}$ is always larger than $N_{AO}$, evaluating (1.12) is actually more expensive than evaluating (1.3).
This leaves the exchange matrix evaluation as the major bottleneck for most density-fitted procedures.
One method for alleviating this bottleneck involves employing knowledge 
of the form of the density matrix. In Hartree-Fock theory, the density matrix is 
\begin{align}
D_{\lambda \sigma} &= C_{\lambda i}C_{\sigma i},
\end{align}

\noindent where $i$ indicates an occupied MO, and there are far fewer occupied MO's than there are AO basis functions.
\noindent Since MO spaces are often much smaller than AO spaces, it is possible to lower the complexity prefactor when 
building the exchange matrix. 
Decomposing the density matrix and rewriting the Coulomb and exchange matrix evaluations, we get:
\begin{align}
J_{\mu \nu} &= B_{\mu \nu}^P B_{\lambda \sigma}^PC_{\lambda i}C_{\sigma i} \\
K_{\mu \nu} &= B_{\mu \lambda}^P B_{\nu \sigma}^PC_{\lambda i}C_{\sigma i}
\end{align}

\noindent Both (1.14) and (1.15) requires $\mathcal{O}(N_{aux}N_{AO}^2N_i)$ operations.
This technique is not beneficial for Coulomb matrix evaluations, since the complexity of (1.14) is actually
higher than that of (1.11). However, (1.15) reduces the complexity of exchange matrix evaluation by  
$\frac{N_{aux}}{N_i}$, where $N_i$ denotes the number of occupied orbitals, which provides moderate speedups in practice.

Moreover, density fitting also improves the cost of integral transformations. Applying (1.10) to (1.5), we get:
\begin{align}
B^P_{pq} = B^P_{\mu \nu}C_{\mu p}C_{\nu q} \\
(pq | rs) \approx B^P_{pq}B^P_{rs}
\end{align}

\noindent The formal computational scaling of this operation is $\mathcal{O}(N_{aux}N_pN_qN_rN_s)$. However, in practice,
algorithms based on density fitting often involve contracting the transformed three-index integrals, $B^P_{pq}$, with other terms, 
rather than explicit formulation of $(pq|rs)$ as actual intermediates. Therefore, optimizing these procedures will primarily
focus on optimizing the computation of $B^P_{pq}$ via (1.16). 

Although density fitting makes considerable progress towards reducing the storage and computation complexity when
using the ERIs, there is still work to be done. Importantly, (1.15) and (1.16) still serve as major computational
bottlenecks for iterative and perturbative methods, respectively. In my research, I have sought out and implemented various 
optimizations to further improve the density fitting regime. The essence of my work involves the union of sparsity screening,
good parallel scaling, minimizing disk I/O, and optimal contraction paths. 
In Chapter 2, I discuss the spatial sparsity of the three-index integrals and the utilization of such sparsity for integral 
construction, metric contractions, and integral transformations.
In Chapter 3, I focus on (1.15) by discussing blocking procedures, sparse memory layouts, disk implications,
and workflow optimizations.
Chapter 4 focuses on (1.14) by formulating exchange matrix builds based on different sparse memory layouts and details 
the scalability of the resulting algorithms on multi-core and manycore architectures.
Finally, Chapter 5 summarizes the results and makes suggestions for future work.

%Chapter 5 discusses a failed attempt at obtaining speedups in second order SCF (SOSCF) procedures by 
%evaluating Coulomb and exchange matrices directly in the MO basis.


