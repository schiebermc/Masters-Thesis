\clearpage
\begin{centering}
\textbf{SUMMARY}\\
\vspace{\baselineskip}
\end{centering}

Density fitting is a rank reduction technique popularly used in quantum chemistry in order to reduce the computational cost 
of evaluating, transforming, and processing the 4-center electron repulsion integrals (ERIs). By utilizing the resolution of the 
identity technique, density fitting reduces the 4-center ERIs into a 3-center form. Doing so not only alleviates the high storage cost
of the ERIs, but it also reduces the computational cost of their involving operations. Still, these operations remain as computational
bottlenecks which commonly plague quantum chemistry procedures. The goal of this thesis is to investigate various optimizations for 
these computational kernels used ubiquitously throughout quantum chemistry. First, we detail the spatial sparsity available to the
3-center integrals and the application of such sparsity to various operations, including integral computation, metric 
contractions, and integral transformations. Next, we investigate sparse memory layouts and their implication 
on the performance of the integral transformation kernel. Next, we analyze two transformation algorithms and how 
their performance will vary depending on the context in which they are used. Then, we propose two sparse memory 
layouts and the resulting performance of Coulomb and exchange evaluations. Since the memory required for these 
tensors grows rapidly, we frame these discussions in the context of their in-core and disk performance. 
We implement these methods in the {\sc Psi4} electronic structure package and reveal that the exchange 
matrix evaluation kernel should vary depending on whether a disk-based implementation must be used.

%\pagenumbering{gobble}  %remove page number on summary page
