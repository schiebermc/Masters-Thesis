\chapter{Evaluating Coulomb and Exchange Matrices}

\section{Coulomb and Exchange Evaluations}

From Chapter 1, we wrote the most efficient evaluations of the three-index ERIs to build the Coulomb and exchange matrices as:

\begin{align}
J_{\mu \nu} &= B_{\mu \nu}^P B_{\lambda \sigma}^PD_{\lambda \sigma} \\
K_{\mu \nu} &= B_{\mu \lambda}^P B_{\nu \sigma}^PC_{\lambda p}C_{\sigma p},
\end{align}

\noindent where (4.1) and (4.2) are evaluated in $\mathcal{O}(N_{AO}^2N_{aux})$ and $\mathcal{O}(N_{AO}^2N_{aux}N_p)$ operations, respectively.
In self-consistent-field methods like Hartree-Fock and density functional theory, the density matrix is constructed by summing over only occupied
molecular orbitals ($N_p=N_{occ}$).
The exchange matrix evaluation is a major bottleneck within the density fitting regime. The focus of this chapter is on analyzing and improving
exchange builds to take advantage of sparsity, optimize parallel scaling, and to minimize disk operations.
Thankfully, it turns out that Algorithm 6 is easily 
extendable to (4.2). In fact, if one uses orbital matrices to build the exchange matrix as in (4.2), the half-transformed 
intermediate generated during integral transformations, $B_{\mu Pq}$, is actually identical to the intermediate when building the exchange matrix. 
Moreover, the memory layout of $B_{\mu Pq}$ is ideal for the subsequent contraction to form $K$. We will consider the possibility that we want to build
"generalized" matrices, where perhaps two different matrices $C$ are used (this is not relevant for self-consistent-field energies, but can arise in 
other circumstances, for example, in SAPT). A generalized Coulomb matrix can be constructed as in (4.1), but where the density matrix is not an SCF density 
matrix, but a generalized density matrix: 

\begin{align}
D_{\lambda \sigma}=C_{\lambda p} C^{\prime}_{\sigma p} 
\end{align}

\noindent and likewise a generalized exchange matrix can be written 

\begin{align}
K_{\mu \nu}= B_{\mu \lambda}^PB_{\nu \sigma}^P C_{\lambda p} C^{\prime}_{\sigma p}.
\end{align}

\noindent Again, the contraction index $p$ does not necessarily run over all MO's, but is typically restricted to some subset, like occupied orbitals.
The following exchange evaluation algorithm results:

\begin{algorithm}[H]
\caption{Building the $K$ matrix.}
\begin{algorithmic}
\REQUIRE Sparse AO integrals: $B_{\mu P \nu^\mu}$, orbital matrices: $C_{\mu p}, C^{\prime}_{\nu p}$, screening mask: $S_{\mu \nu}^b$
\FOR {$\mu = 0$ to $\mu = N_{AO}-1$}  
    \STATE Trim from dense to sparse: $C_{\nu p}S_{\mu \nu}^b \rightarrow C_{\nu^{\mu} p}$
    \STATE $B_{\mu P \nu^{\mu}} C_{\nu^{\mu} p} \rightarrow T_{\mu Pp}$
    \IF {$C^{\prime}_{\nu p} != C_{\nu p}$}
        \STATE Trim from dense to sparse: $C^{\prime}_{\nu p}S_{\mu \nu}^b \rightarrow C^{\prime}_{\nu^{\mu} p}$
        \STATE $B_{\mu P \nu^{\mu}} C^{\prime}_{\nu^{\mu} p} \rightarrow T^{\prime}_{\mu Pp}$
    \ELSE
        \STATE $T_{\mu P p}^{\prime} = T_{\mu P p}$ 
    \ENDIF
\ENDFOR
\STATE $T_{\mu P p} T_{\nu P p}^{\prime} \rightarrow K_{\mu \nu} $
\RETURN $K_{\mu \nu}$
\end{algorithmic}
\end{algorithm}

We have used $T$ tensors to indicate intermediates.
Algorithm 9 is an ideal candidate for building the $K$ matrix, as it both utilizes sparsity and maximizes concurrency.
However, while it may be assumed that both $J$ and $K$ can always fit within in-core memory constraints, the same is not true 
for the three-index AO integrals. Therefore, we must consider the behavior of Algorithm 9 when the 
in-core memory is constrained such that it becomes necessary to perform blocking operations across the $P$ index. Then, the algorithm
becomes: 

\begin{algorithm}[H]
\caption{Building the $K$ matrix using $B_{\mu P \nu^\mu}$, blocking across $P$}
\begin{algorithmic}
\REQUIRE Sparse AO integrals: $B_{\mu P \nu^\mu}$, orbital matrices: $C_{\mu p}, C^{\prime}_{\nu p}$, screening mask: $S_{\mu \nu}^b$
\FOR {block $P_i \in \{P\}$}    
    \STATE Read from disk: $B_{\mu P_i \nu^{\mu}}$
    \FOR {$\mu = 0$ to $\mu = N_{AO}-1$}  
        \STATE Trim from dense to sparse: $C_{\nu p}S_{\mu \nu}^b \rightarrow C_{\nu^{\mu} p}$
        \STATE $B_{\mu P \nu^{\mu}} C_{\nu^{\mu} p} \rightarrow T_{\mu Pp}$
        \IF {$C^{\prime}_{\nu p} != C_{\nu p}$}
            \STATE Trim from dense to sparse: $C^{\prime}_{\nu p}S_{\mu \nu}^b \rightarrow C^{\prime}_{\nu^{\mu} p}$
            \STATE $B_{\mu P \nu^{\mu}} C^{\prime}_{\nu^{\mu} p} \rightarrow T^{\prime}_{\mu Pp}$
        \ELSE
            \STATE $T_{\mu P p}^{\prime} = T_{\mu P p}$ 
        \ENDIF
    \ENDFOR
    \STATE $K_{\mu \nu} = K_{\mu \nu} + T_{\mu P_i p} T^{\prime}_{\nu P_i p}$
\ENDFOR
\RETURN $K_{\mu \nu}$
\end{algorithmic}
\end{algorithm}

\noindent Unfortunately, this algorithm will require strided disk operations when reading the $B_{\mu P^i \nu^{\mu}}$ tensor into memory.
Since poor disk I/O can drastically decrease performance, it is often better to force disk operations 
to be contiguous and transpose any tensors as necessary in core memory. Since it is best to block across the $P$ index (as discussed in
Chapter 3), a better
memory layout for the sparse integrals is the $B_{P \mu \nu^\mu}$ form, since this form will allow for completely 
contiguous reads from disk memory. Another possible algorithm can then be formulated:

\begin{algorithm}[H]
\caption{Building the $K$ matrix using $B_{P \mu \nu^\mu}$, blocking across $P$}
\begin{algorithmic}
\REQUIRE Sparse AO integrals: $B_{P \mu \nu^\mu}$, orbital matrices: $C_{\mu p}, C^{\prime}_{\nu p}$, screening mask: $S_{\mu \nu}^b$
\FOR {block $P_i \in \{P\}$}
    \STATE Read from disk: $B_{P_i \mu \nu^{\mu}}$
    \FOR {$\mu = 0$ to $\mu = N_{AO}-1$}  
        \STATE Copy: $B_{P_i \mu \nu^{\mu}} \rightarrow b_{P_i \nu^{\mu}}$
        \STATE Trim from dense to sparse: $C_{\nu p}S_{\mu \nu}^b \rightarrow C_{\nu^{\mu} p}$
        \STATE $b_{P_i \nu^{\mu}} C_{\nu^{\mu} p} \rightarrow T_{\mu P_i p}$
        \IF {$C^{\prime}_{\nu p} != C_{\nu p}$}
            \STATE Trim from dense to sparse: $C^{\prime}_{\nu p}S_{\mu \nu}^b \rightarrow C^{\prime}_{\nu^{\mu} p}$
            \STATE $b_{P_i \nu^{\mu}} C_{\nu^{\mu} p} \rightarrow T^{\prime}_{\mu P_i p}$
        \ELSE
            \STATE $T^{\prime}_{\mu P_i p} = T_{\mu P_i p}$ 
        \ENDIF
    \ENDFOR
    \STATE $K_{\mu \nu} = K +  T_{\mu P_i p} T^{\prime}_{\nu P_i p} $
\ENDFOR
\RETURN $K_{\mu \nu}$
\end{algorithmic}
\end{algorithm}

\noindent Here, we used a buffer, $b_{P^i \nu^{\mu}}$, to transpose pieces of $B_{P^i \mu \nu^{\mu}}$ while looping over $\mu$.
Although this algorithm may involve a strided copy and additional memory usage, the disk reads for the $B_{P^i \mu \nu^{\mu}}$ 
tensor are completely contiguous. 
For smaller investigations, the three-index AO integrals can be fit completely in-core and Algorithm 10 should yield superior 
performance. However, for larger systems,
the strided disk reads in Algorithm 10 may cause performance degradation to the point that Algorithm 11 will become superior.

Although the exchange matrix evaluation is the primary computational bottleneck, it is important to note how the above integral
formats will affect Coulomb matrix evaluations. To build the Coulomb matrix, the sparsity mask can be applied
directly to the density matrix. The following algorithm results:

\begin{algorithm}[H]
\caption{Building the $J$ matrix.}
\begin{algorithmic}
\REQUIRE Sparse AO integrals: $B_{\mu \nu^{\mu}}^P$, density matrix: $D_{\mu \nu}$, screening mask: $S_{\mu \nu}^b$
\STATE Trim from dense to sparse: $D_{\mu \nu}S_{\mu \nu}^b \rightarrow D_{\mu \nu^{\mu} }$
\STATE $B^P_{\mu\nu^{\mu}} D_{\mu \nu^{\mu}} \rightarrow T^{P}$
\STATE $T^{P} B^P_{\mu \nu^{\mu}} \rightarrow J_{\mu \nu^{\mu}} $
\STATE Unpack from sparse to dense: $J_{\mu \nu^{\mu}} \rightarrow J_{\mu \nu}$
\RETURN $J_{\mu \nu}$
\end{algorithmic}
\end{algorithm}

\noindent Moreover, the corresponding disk-based implementations for the $B_{\mu P \nu^\mu}$ and $B_{P \mu \nu^\mu}$ integral
formats are listed in Algorithms 13 and 14, respectively.

\begin{algorithm}[H]
\caption{Building the $J$ matrix using $B_{\mu P \nu^\mu}$, blocking across $P$}
\begin{algorithmic}
\REQUIRE Sparse AO integrals: $B_{\mu P \nu^{\mu}}$, density matrix: $D_{\mu \nu}$, screening mask: $S_{\mu \nu}^b$
\FOR {block $P_i \in P$}
    \STATE Read from disk: $B_{\mu P_i \nu^{\mu}}$
    \STATE Initialize: $T_{P_i} = 0$
    \FOR {$\mu = 0$ to $\mu = N_{AO}-1$}  
        \STATE Copy to sparse: $D_{\mu \nu}S_{\mu \nu}^b \rightarrow d_{\nu^{\mu} }$
        \STATE $T_{P_i} = T_{P_i} + B_{\mu P_i \nu^{\mu}}d_{\nu^{\mu}}$
    \ENDFOR    
    \STATE $T_{P_i} B_{P_i \mu \nu^{\mu}} \rightarrow J_{\mu \nu^{\mu}} $
    \STATE Unpack from sparse to dense: $J_{\mu \nu^{\mu}} \rightarrow J_{\mu \nu}^{\{P_i\}}$
    \STATE $J_{\mu \nu} = J_{\mu \nu} + J_{\mu \nu}^{\{P_i\}}$
\ENDFOR
\RETURN $J_{\mu \nu}$
\end{algorithmic}
\end{algorithm}

\begin{algorithm}[H]
\caption{Building the $J$ matrix using $B_{P \mu \nu^\mu}$, blocking across $P$}
\begin{algorithmic}
\REQUIRE Sparse AO integrals: $B_{P \mu \nu^{\mu}}$, density matrix: $D_{\mu \nu}$, screening mask: $S_{\mu \nu}^b$
\FOR {block $P_i \in P$}
    \STATE Read from disk: $B_{P_i \mu \nu^{\mu}}$
    \STATE Trim from dense to sparse: $D_{\mu \nu}S_{\mu \nu}^b \rightarrow D_{\mu \nu^{\mu} }$
    \STATE $B_{P_i \mu\nu^{\mu}} D_{\mu \nu^{\mu}} \rightarrow T^{P_i}$
    \STATE $T^{P_i} B_{P_i \mu \nu^{\mu}} \rightarrow J_{\mu \nu^{\mu}} $
    \STATE Unpack from sparse to dense: $J_{\mu \nu^{\mu}} \rightarrow J_{\mu \nu}^{\{P_i\}}$
    \STATE $J_{\mu \nu} = J_{\mu \nu} + J_{\mu \nu}^{\{P_i\}}$
\ENDFOR
\RETURN $J_{\mu \nu}$
\end{algorithmic}
\end{algorithm}

\noindent Here, we have used the superscript in $J_{\mu \nu}^{\{P_i\}}$ to indicate the contribution to 
$J_{\mu \nu}$ corresponding to the $P_i$ block.
Note that Algorithm 13 is likely inferior to Algorithm 14, with both necessary loops and copies as well as non-contiguous disk reads.
However, since Coulomb matrix evaluation requires considerably less compute time than the exchange matrix evaluations, 
the time required by Algorithms 13 and 14 will be nearly trivial compared to the exchange form. 
Moreover, note that the disk reads in Algorithms 10 and 13 as well as 
in Algorithms 11 and 14 will be occurring simultaneously. The $J$ and $K$ computational kernel will generally take this form:

\begin{algorithm}[H]
\caption{Coulomb and exchange matrix evaluation kernel.}
\begin{algorithmic}
\REQUIRE Sparse AO integrals: $B^P_{\mu \nu^{\mu}}$, density matrix: $D_{\mu \nu}$, orbital matrices: $C_{\mu p}, C^{\prime}_{\nu p}$, screening mask: $S_{\mu \nu}^b$
\STATE Initialize temps.
\FOR {block $P_i \in P$}
    \STATE Read from disk: $B^{P_i}_{\mu \nu^{\mu}}$
    \STATE $J_{\mu \nu} = J_{\mu \nu}$ + ComputeJ($B^{P_i}_{\mu \nu^{\mu}}$, $S_{\mu \nu}^b$, $D_{\mu \nu}$)
    \STATE $K_{\mu \nu} = K_{\mu \nu}$ + ComputeK($B^{P_i}_{\mu \nu^{\mu}}$, $S_{\mu \nu}^b$, $C_{\mu p}, C^{\prime}_{\nu p}$)
\ENDFOR
\RETURN $J_{\mu \nu}$, $K_{\mu \nu}$
\end{algorithmic}
\end{algorithm}

\noindent An implementation of this kernel will involve using the $B_{\mu P \nu^\mu}$ integral form with Algorithms 10 and 13 or the 
$B_{\mu P \nu^\mu}$ integral form with Algorithms 11 and 14. In the next section, we discuss the performance of these choices in practice.

\section{Results}

All methods were implemented in the {\sc Psi4} electronic structure software package \cite{Parrish:2017:3185}.
The parallelism in {\sc Psi4} relies on the shared memory programming model using OpenMP 
and carries out matrix multiplications using Intel's Math Kernel
Library. Currently, the state of the art for exchange matrix evaluations in
{\sc Psi4} is Algorithm 11. However, we conjecture that Algorithm 10 could provide 
considerable speedups as it eliminates entirely a strided, level 1 BLAS copy. 

\subsection{Multi-core Results}

First, we reveal the performance of the above algorithms on multi-core processors.
We implemented Algorithms 10 and 13 and incorporated them into a development version of {\sc Psi4}. 
Then, we used {\sc Psi4}'s Self-Consistent Field
procedure to produce energies for various systems and basis set combinations. The systems used involved a protein-drug complex,
where the drug molecule is omitted and the atoms of the protein are added in a series according to distance from the center of
the drug molecule. 

The experiments were carried out on two separate architectures. The first includes one node consisting of an Intel Core i7-5930K processor
(6 cores at 3.50GHz) and 64GB DRAM. The results for this architecture are included in Table 4.1. The second architecture includes one node consisting
of two Intel Xeon E5-2640 processors and 256GB DRAM. The results for this architecture are included in Table 4.2. 

\begingroup
%\squeezetable
\renewcommand{\arraystretch}{0.6}
\begin{table}[H]
\footnotesize
\centering
\renewcommand{\baselinestretch}{1}
\caption{Total execution times for Self-Consistent Field procedures for various systems using Algorithms 10 and 11 and Algorithms 13 and 14 for
exchange and Coulomb matrix evaluations, respectively. Total wall times include 
wall time for the entire program to execute. J and K compute time indicates the total time spent in the Coulomb and exchange matrix
evaluation kernels, respectively. System descriptions including number of AO basis functions, $N_{AO}$, number of auxiliary basis functions,
$N_{aux}$, basis, and number of atoms, $N_{at.}$, are included. All rows are sorted by the product $N_{aux}N_{AO}$.
A speedup column is included for total wall time and is calculated as the total procedure time spent using Algorithms 11 and 14 dived by 
the total procedure time spent using Algorithms 10 and 13. The experiments were carried out using one node consisting of an Intel Core i7-5930K processor
(6 cores at 3.50GHz) and 64GB DRAM
\label{tbl:practical_speedups}}
\begin{tabular}{lrrrrrrrrrr}
  \multicolumn{1}{c}{\textbf{}} 
& \multicolumn{1}{c}{\textbf{}} 
& \multicolumn{1}{c}{\textbf{}} 
& \multicolumn{1}{c}{\textbf{}} 
& \multicolumn{3}{c}{\textbf{Total Wall Time}}  
& \multicolumn{2}{c}{\textbf{J Compute Time}}  
& \multicolumn{2}{c}{\textbf{K Compute Time}} \\ 
\cline{5-7}
\cline{8-9}
\cline{10-11}
  \multicolumn{1}{c}{\textbf{$N_{AO}$}} 
& \multicolumn{1}{c}{\textbf{$N_{aux}$}} 
& \multicolumn{1}{c}{\textbf{Basis}} 
& \multicolumn{1}{c}{\textbf{$N_{at.}$}} 
& \multicolumn{1}{c}{\textbf{10 \& 13}} 
& \multicolumn{1}{c}{\textbf{11 \& 14}} 
& \multicolumn{1}{c}{\textbf{spdup}} 
& \multicolumn{1}{c}{\textbf{Alg. 14}} 
& \multicolumn{1}{c}{\textbf{Alg. 13}} 
& \multicolumn{1}{c}{\textbf{Alg. 10}} 
& \multicolumn{1}{c}{\textbf{Alg. 11}} \\ 
\hline
 147&  721&    DZ&    15&                 3.8 &                4.7&     1.2 &                0.1 &                0.0&                 0.5&                 0.6\\
 247&  912&   aDZ&    15&                 5.8 &                7.7&     1.3 &                0.2 &                0.1&                 1.4&                 2.1\\
 338&  842&    TZ&    15&                 7.5 &                9.9&     1.3 &                0.2 &                0.2&                 2.3&                 3.3\\
 294& 1442&    DZ&    30&                12.4 &               18.1&     1.5 &                0.3 &                0.3&                 5.6&                 7.1\\
 529& 1154&   aTZ&    15&                18.7 &               29.9&     1.6 &                0.8 &                0.8&                 7.4&                13.1\\
 375& 1837&    DZ&    39&                25.4 &               36.8&     1.4 &                0.5 &                0.5&                13.3&                16.6\\
 650& 1205&    QZ&    15&                24.7 &               43.8&     1.8 &                1.2 &                1.1&                11.1&                19.3\\
 494& 1824&   aDZ&    30&                35.3 &               54.8&     1.6 &                1.0 &                0.9&                18.1&                25.7\\
 676& 1684&    TZ&    30&                46.0 &               74.7&     1.6 &                1.2 &                1.2&                28.2&                38.6\\
 522& 2558&    DZ&    54&                72.3 &              109.2&     1.5 &                1.1 &                0.9&                48.8&                56.8\\
 631& 2328&   aDZ&    39&                81.5 &              132.9&     1.6 &                2.1 &                2.0&                46.8&                65.1\\
 603& 2953&    DZ&    63&               117.2 &              166.9&     1.4 &                1.3 &                1.1&                83.2&                92.4\\
 866& 2150&    TZ&    39&               109.3 &              176.6&     1.6 &                2.5 &                2.4&                74.7&                97.6\\
1058& 2308&   aTZ&    30&               145.5 &              278.9&     1.9 &                4.6 &                4.7&                91.0&               138.6\\
 756& 3696&    DZ&    81&               231.3 &              328.3&     1.4 &                1.8 &                1.6&               176.3&               190.9\\
 878& 3240&   aDZ&    54&               232.0 &              394.4&     1.7 &                4.6 &                4.4&               161.9&               206.3\\
1300& 2410&    QZ&    30&               218.0 &              376.1&     1.7 &                5.6 &                5.8&               148.7&               198.3\\
1204& 2992&    TZ&    54&               332.5 &              504.1&     1.5 &                4.9 &                4.7&               256.8&               302.4\\
1015& 3744&   aDZ&    63&               403.1 &              640.8&     1.6 &                6.3 &                6.1&               305.2&               361.9\\
1357& 2954&   aTZ&    39&               385.9 &              726.9&     1.9 &               10.5 &               12.3&               259.2&               376.6\\
 974& 4766&    DZ&   103&               629.3 &              845.3&     1.3 &                3.6 &                3.2&               520.0&               549.7\\
1394& 3458&    TZ&    63&               601.7 &              850.5&     1.4 &                6.6 &                6.2&               492.8&               551.2\\
1670& 3089&    QZ&    39&  1153.5$^{\dagger}$ &              933.4&     0.8 &   12.8$^{\dagger}$ &               13.2&   395.5$^{\dagger}$&               504.7\\
1275& 4698&   aDZ&    81&               898.9 &             1379.0&     1.5 &               10.6 &               10.7&               711.5&               813.2\\
1758& 4341&    TZ&    81&              1372.3 &             1839.0&     1.3 &               10.2 &                9.8&              1176.8&              1269.1\\
1886& 4108&   aTZ&    54&  3811.0$^{\dagger}$ &             2164.0&     0.6 &   25.8$^{\dagger}$ &               26.2&   931.2$^{\dagger}$&              1183.4\\
1641& 6050&   aDZ&   103&  4406.9$^{\dagger}$ &             3528.3&     0.8 &   23.9$^{\dagger}$ &               24.6&  1958.4$^{\dagger}$&              2161.9\\
2320& 4294&    QZ&    54&  4280.4$^{\dagger}$ &             2741.3&     0.6 &   26.3$^{\dagger}$ &               25.0&  1377.2$^{\dagger}$&              1611.1\\
2185& 4754&   aTZ&    63&  5744.9$^{\dagger}$ & 4594.0$^{\dagger}$&     0.8 &   37.3$^{\dagger}$ &   37.0$^{\dagger}$&  1727.3$^{\dagger}$&  2063.1$^{\dagger}$\\
2258& 5589&    TZ&   103&  5474.8$^{\dagger}$ &             4710.6&     0.9 &   23.5$^{\dagger}$ &               20.2&  3228.0$^{\dagger}$&              3381.7\\
2690& 4973&    QZ&    63&  6892.9$^{\dagger}$ &             4581.9&     0.7 &   39.5$^{\dagger}$ &               33.7&  2598.4$^{\dagger}$&              2893.4\\
2760& 5988&   aTZ&    81& 11294.6$^{\dagger}$ & 9350.0$^{\dagger}$&     0.8 &   64.6$^{\dagger}$ &   63.4$^{\dagger}$&  4003.6$^{\dagger}$&  4669.9$^{\dagger}$\\
3405& 6276&    QZ&    81& 14075.7$^{\dagger}$ &11354.3$^{\dagger}$&     0.8 &   70.6$^{\dagger}$ &   54.7$^{\dagger}$&  6318.9$^{\dagger}$&  6829.2$^{\dagger}$\\
3542& 7696&   aTZ&   103& 28136.3$^{\dagger}$ &23132.6$^{\dagger}$&     0.8 &  144.4$^{\dagger}$ &  124.7$^{\dagger}$& 11201.2$^{\dagger}$& 12464.4$^{\dagger}$\\

\hline
\end{tabular}
\renewcommand{\thefootnote}{\fnsymbol{footnote}}
\footnote[2]{} Indicates a disk-based implementation was used.
\end{table}
\endgroup


\begingroup
%\squeezetable
\renewcommand{\arraystretch}{0.6}
\begin{table}[H]
\footnotesize
\centering
\renewcommand{\baselinestretch}{1}
\caption{Total execution times for Self-Consistent Field procedures for various systems using Algorithms 10 and 11 and Algorithms 13 and 14 for
exchange and Coulomb matrix evaluations, respectively. Total wall times include 
wall time for the entire program to execute. J and K compute time indicates the total time spent in the Coulomb and exchange matrix
evaluation kernels, respectively. System descriptions including number of AO basis functions, $N_{AO}$, number of auxiliary basis functions,
$N_{aux}$, basis, and number of atoms, $N_{at.}$, are included. All rows are sorted by the product $N_{aux}N_{AO}$.
A speedup column is included for total wall time and is calculated as the total procedure time spent using Algorithms 11 and 14 dived by 
the total procedure time spent using Algorithms 10 and 13. The experiments were carried out using one node consisting
of two Intel Xeon E5-2640 processors (10 cores at 2.40GHz) and 256GB DRAM.
\label{tbl:practical_speedups}}
\begin{tabular}{lrrrrrrrrrr}
  \multicolumn{1}{c}{\textbf{}} 
& \multicolumn{1}{c}{\textbf{}} 
& \multicolumn{1}{c}{\textbf{}} 
& \multicolumn{1}{c}{\textbf{}} 
& \multicolumn{3}{c}{\textbf{Total Wall Time}}  
& \multicolumn{2}{c}{\textbf{J Compute Time}}  
& \multicolumn{2}{c}{\textbf{K Compute Time}} \\ 
\cline{5-7}
\cline{8-9}
\cline{10-11}
  \multicolumn{1}{c}{\textbf{$N_{AO}$}} 
& \multicolumn{1}{c}{\textbf{$N_{aux}$}} 
& \multicolumn{1}{c}{\textbf{Basis}} 
& \multicolumn{1}{c}{\textbf{$N_{at.}$}} 
& \multicolumn{1}{c}{\textbf{10 \& 13}} 
& \multicolumn{1}{c}{\textbf{11 \& 14}} 
& \multicolumn{1}{c}{\textbf{spdup}} 
& \multicolumn{1}{c}{\textbf{Alg. 13}} 
& \multicolumn{1}{c}{\textbf{Alg. 14}} 
& \multicolumn{1}{c}{\textbf{Alg. 10}} 
& \multicolumn{1}{c}{\textbf{Alg. 11}} \\ 
\hline

 147 & 721&    DZ& 15&                 2.9&                 3.4 & 1.2 &                0.1 &                0.0 &                0.1 &                0.2\\  
 247 & 912&   aDZ& 15&                 3.8&                 5.2 & 1.4 &                0.2 &                0.2 &                0.4 &                0.7\\
 338 & 842&    TZ& 15&                 5.0&                 6.5 & 1.3 &                0.2 &                0.3 &                0.6 &                1.1\\
 294 &1442&    DZ& 30&                 7.5&                 9.4 & 1.3 &                0.2 &                0.2 &                1.1 &                1.6\\
 529 &1154&   aTZ& 15&                 8.8&                13.4 & 1.5 &                0.7 &                0.9 &                1.7 &                3.7\\
 375 &1837&    DZ& 39&                10.5&                17.7 & 1.7 &                0.4 &                0.5 &                2.5 &                4.1\\
 650 &1205&    QZ& 15&                11.4&                18.7 & 1.6 &                0.9 &                1.1 &                2.2 &                4.7\\
 494 &1824&   aDZ& 30&                14.7&                22.9 & 1.6 &                0.8 &                1.1 &                3.6 &                6.5\\
 676 &1684&    TZ& 30&                17.3&                30.6 & 1.8 &                1.2 &                1.3 &                5.1 &                8.4\\
 522 &2558&    DZ& 54&                21.8&                39.3 & 1.8 &                0.9 &                1.0 &                7.4 &               10.6\\
 631 &2328&   aDZ& 39&                27.2&                50.2 & 1.8 &                1.8 &                2.2 &                8.3 &               16.4\\
 962 &1668&   aQZ& 15&                27.2&                45.5 & 1.7 &                2.8 &                3.6 &                6.6 &               15.9\\
 603 &2953&    DZ& 63&                29.1&                57.0 & 2.0 &                1.1 &                1.2 &               11.4 &               16.2\\
 866 &2150&    TZ& 39&                34.5&                60.5 & 1.8 &                2.0 &                2.6 &               11.0 &               19.6\\
1058 &2308&   aTZ& 30&                45.5&                86.8 & 1.9 &                3.8 &                5.2 &               15.6 &               34.2\\
 756 &3696&    DZ& 81&                50.4&               102.6 & 2.0 &                1.6 &                1.8 &               23.1 &               30.5\\
 878 &3240&   aDZ& 54&                62.4&               119.8 & 1.9 &                3.6 &                5.0 &               25.0 &               45.7\\
1300 &2410&    QZ& 30&                58.4&               105.1 & 1.8 &                4.4 &                5.8 &               20.6 &               40.1\\
1204 &2992&    TZ& 54&                71.0&               133.6 & 1.9 &                3.8 &                5.0 &               32.8 &               51.4\\
1015 &3744&   aDZ& 63&                88.0&               169.5 & 1.9 &                4.6 &                5.9 &               38.1 &               64.0\\
1357 &2954&   aTZ& 39&                96.2&               181.1 & 1.9 &                7.7 &               11.4 &               36.8 &               81.7\\
 974 &4766&    DZ&103&               112.8&               226.0 & 2.0 &                3.3 &                3.4 &               60.1 &               78.2\\
1394 &3458&    TZ& 63&               103.2&               217.8 & 2.1 &                4.4 &                6.1 &               50.9 &               75.9\\
1670 &3089&    QZ& 39&               119.9&               221.4 & 1.8 &                7.8 &               12.3 &               50.0 &               96.9\\
1275 &4698&   aDZ& 81&               175.7&               345.9 & 2.0 &                7.6 &               11.1 &               86.3 &              135.9\\
1924 &3336&   aQZ& 30&               184.9&               346.0 & 1.9 &               14.5 &               25.0 &               67.4 &              154.0\\
1758 &4341&    TZ& 81&               207.3&               386.2 & 1.9 &                7.2 &               10.0 &              122.3 &              164.5\\
1886 &4108&   aTZ& 54&               262.0&               471.2 & 1.8 &               14.7 &               25.8 &              117.9 &              221.8\\
1641 &6050&   aDZ&103&               416.8&               798.6 & 1.9 &               13.8 &               23.3 &              231.3 &              352.0\\
2320 &4294&    QZ& 54&               328.4&               646.8 & 2.0 &               18.9 &               26.2 &              173.0 &              273.0\\
2185 &4754&   aTZ& 63&               399.4&               704.1 & 1.8 &               23.4 &               33.2 &              195.4 &              322.9\\
2474 &4284&   aQZ& 39&               481.1&               814.1 & 1.7 &               35.4 &               50.0 &              171.6 &              395.6\\
2258 &5589&    TZ&103&               495.3&               884.8 & 1.8 &               15.1 &               18.8 &              329.3 &              420.6\\
2690 &4973&    QZ& 63&               495.4&               837.5 & 1.7 &               22.0 &               30.8 &              265.5 &              389.7\\
%2760 &5988&   aTZ& 81& 20919.0$^{\dagger}$&              1913.1 & 0.1 &  293.0$^{\dagger}$ &               41.0 & 1134.6$^{\dagger}$ &              641.3\\
%3436 &5952&   aQZ& 54& 48091.4$^{\dagger}$&              1932.8 & 0.0 &  917.0$^{\dagger}$ &               75.9 & 2082.7$^{\dagger}$ &              959.2\\
3405 &6276&    QZ& 81&              1042.8&              1713.8 & 1.6 &               42.9 &               42.0 &              598.9 &              814.3\\
%3542 &7696&   aTZ&103& 51741.8$^{\dagger}$&              3307.3 & 0.1 &  508.8$^{\dagger}$ &               82.9 & 3628.6$^{\dagger}$ &             1768.0\\
%3986 &6900&   aQZ& 63& 67706.5$^{\dagger}$& 32182.2$^{\dagger}$ & 0.5 & 2005.0$^{\dagger}$ &  125.0$^{\dagger}$ & 4666.8$^{\dagger}$ & 1656.0$^{\dagger}$\\
%4365 &8058&    QZ&103& 46951.0$^{\dagger}$&              4862.1 & 0.1 & 1505.7$^{\dagger}$ &               68.6 & 5515.4$^{\dagger}$ &             3000.2\\

\hline
\end{tabular}
\renewcommand{\thefootnote}{\fnsymbol{footnote}}
\footnote[2]{} Indicates a disk-based implementation was used.
\end{table}
\endgroup

Note that the $^{\dagger}$ symbol is used to indicate when a disk-based implementation is required. For Algorithm 11, it is possible
to store up to half of the AO function pairs by applying permutational symmetry. The necessary copy, 
$B_{P_i \mu \nu^{\mu}} \rightarrow b_{P_i \nu^{\mu}}$, allows for this. Doing so effectively halves the memory requirement for Algorithm 11
with respect to Algorithm 10. For this reason, Algorithm 10 is forced to resort to a disk-based implementation sooner than Algorithm 11.

The results in Table 4.1 and Table 4.2 confirm our analysis of Algorithms 10 and 11. For small enough investigations, procedures using Algorithm 10 will always be more
efficient. Total time spent in the exchange matrix evaluation kernel is always smaller for Algorithm 10. 
Moreover, the speedups obtained on the two-processor architecture (Table 4.2) are larger than what was obtained on the one-processor architecture (Table 4.1).
This result reveals the importance of memory locality and data movement on non-uniform memory access (NUMA) architectures. Algorithm 11 requires a strided level 1 BLAS operation and 
this additional data movement especially hinders performance on NUMA architectures.
Only when the procedures switch to a disk-based implementation
do strided reads of the $B_{\mu P \nu^\mu}$ tensor result in a performance switch when using the two integral memory layouts. 
Moreover, since the scaling of the Coulomb matrix evaluation is an entire factor smaller than the exchange matrix evaluation, its required time is almost trivial overall.

Lastly, note that the total time spent in the $J$ and $K$ evaluations does not entirely account for the differences in performance. The 
remaining difference involves the computation of $B^P_{\mu \nu^\mu}$, which with a metric contraction scaling as 
$\mathcal{O}(N_{aux}^2N_{AO}^2)$, can be the most expensive operation of an SCF procedure. Even though Algorithm 10 does not 
apply permutational symmetry, it can be more efficient for this operation as both the contractions and disk writes are contiguous.


\subsection{Manycore Results}

Here we examine the performance of the above algorithms on Intel's Knights Landing manycore processor. 
We implemented Algorithms 10 and 13 and incorporated them into a development version of {\sc Psi4}. 
Then, we used {\sc Psi4}'s Self-Consistent Field procedure to produce energies for the system in Figure 3.1 using the cc-pVTZ basis set. 
The experiments were carried out using one node consisting of one Knights Landing processor. Procedures were repeated using varying numbers of cores.
Tables 4.3 and 4.4 include total procedure times and exchange matrix evaluation times, respectively.   

\begingroup
\begin{table}[H]
\centering
\renewcommand{\baselinestretch}{1}
\caption{Total procedure execution times for a Self-Consistent Field procedure using both Algorithms 10 and 13 and Algorithms 11 and 14 for exchange and Coloumb matrix evaluations, 
    respectively. The programs were executed on one Knights Landing processor
    using a development version of {\sc Psi4}. A relative speedup is listed.}
\begin{tabular}{l ccc}
\multicolumn{1}{l}{\textbf{Cores}} &
\multicolumn{1}{c}{\textbf{10 \& 13}} &
\multicolumn{1}{c}{\textbf{11 \& 14}} &
\multicolumn{1}{c}{\textbf{Speedup}} \\
\hline

1   &3881.67&   8828.52 &2.3 \\
2   &2278.7 &   4833.24 &2.1 \\
4   &1289.95&   2454.99 &1.9 \\
8   &703.25&    1290.01 &1.8 \\
16  &421.24&    735.1   &1.7 \\
32  &302.46&    507.31  &1.7 \\
64  &280.56&    467.03  &1.7 \\

\end{tabular}
\end{table}
\endgroup


\begingroup
\begin{table}[H]
\centering
\renewcommand{\baselinestretch}{1}
\caption{Total exchange matrix evaluation times within an Self-Consistent Field procedure using both Algorithms 10 and 11. The programs were executed on one Knights Landing processor
    using a development version of {\sc Psi4}. A relative speedup is listed.}
\begin{tabular}{l ccc}
\multicolumn{1}{l}{\textbf{Cores}} &
\multicolumn{1}{c}{\textbf{Alg. 10}} &
\multicolumn{1}{c}{\textbf{Alg. 11}} &
\multicolumn{1}{c}{\textbf{Speedup}} \\
\hline

1   &1695.519&  4217.851&   2.5\\
2   & 979.497&  2368.849&   2.4\\
4   & 477.094&  1159.278&   2.4\\
8   & 242.393&   572.458&   2.4\\
16  & 125.677&   291.958&   2.3\\
32  &  73.627&   188.214&   2.6\\
64  &  64.841&   180.905&   2.8\\

\end{tabular}
\end{table}
\endgroup 

Again, Algorithm 10 is shown to be superior to Algorithm 11 for in-core procedures. Tables 4.3 and 4.4 reveal the most substantial speedups yet, which can be attributed
to the increased NUMA issues for Algorithm 11 on the Knights Landing processor. Moreover, the parallel scaling of  Algorithm 10 is superior to Algorithm 11. However,
the implementation using Algorithms 10 and 13 is shown to scale worse overall.  


